\documentclass{article}
\usepackage[utf8]{inputenc}

\usepackage{tabularx}
\usepackage{graphicx}
\usepackage{adjustbox}
\usepackage{hyperref}
\usepackage{etoolbox}
\usepackage{changepage}
\usepackage{float}

\usepackage{todonotes}
%use this line instaed to hide todo's
%\usepackage[disable]{todonotes}

\title{Halbleiterphysik Fragenkatalog}
\author{Constantin Schieber\\Benedikt Tutzer}
\date{August 2020}

%\patchcmd{\subsubsection}{\normalsize}{\normalsize}{\typeout{same line subsec f}}{\typeout{same line subsec failed}}%
%\patchcmd{\subsubsection}{\bfseries}{}{\typeout{same line subsec f}}{\typeout{same line subsec failed}}%

\newtoggle{aftersection}
\preto{\section}{\filbreak\global\toggletrue{aftersection}}
\preto{\subsection}{\iftoggle{aftersection}{\global\togglefalse{aftersection}}{\filbreak}}
\newcommand{\clearpageafterfirst}{%
  \gdef\clearpageafterfirst{\clearpage}%
}


\begin{document}


\maketitle
\vfill
Diese Ausarbeitung ist zur Vorbereitung auf die m\"undliche Pr\"ufung im August 2020 entstanden. Sie kann gern auf overleaf aktuell gehalten werden.\\

\begin{center}\href{Overleaf public link}{https://www.overleaf.com/4852512811wdbqyxrbkgyb}\end{center}

\setcounter{section}{-1}
\newpage
\tableofcontents
\newpage

%--------------------------------------------------------------------------------
%--------------------------------------------------------------------------------
%--------------------------------------------------------------------------------
\section{Overview}

\begin{center}
\begin{table}[H]
%\begin{adjustwidth}{-3cm}{-3cm}
\begin{adjustbox}{width=\textwidth}
\centering
\begin{tabular}{lcccccccc}
Frage                & 02.16 & 01.28 & 02.19 & 03.04 & 03.05 & 03.21 & 01.16 & Katalog\\
                     & 2010  & 2014  & 2019  & 2019  & 2019  & 2019  & 2020 & \\
                     \hline
\ref{k1:photoEf} Photo-Effekt &&&&&&&& X \\
\ref{k1:comptonEf} Compton-Effekt &&&&&&&& X \\
\ref{k1:dispersionsrelation} Dispersionsrelation für Halbleiter && X &&& X & X & X & \\
\ref{k1:schrGl} Schrödinger Gleichung&&&& X &  & X & X & X \\
\ref{k1:ekdiag} e(k) Diagramm &&&&&&&& X \\
\ref{k1:pottopf} Unendlich tiefer Potentialtopf & X &&&&&&& X \\
\ref{k1:tunnEf} Tunneleffekt&&&&&&&X & \\
\ref{k2:metalle} Unterschied Halbleiter vs Metalle &&&&&&&& X\\
\ref{k2:festkorper} Grund für Festkörperbildung & X &&&&&&& X\\
\ref{k2:leitungsBand} Leitungsband, Valenzband, Ef aufzeichnen && X & X &&&&& \\
\ref{k2:kroningpenny} Kroning Penny Modell &&&&&&&& X \\
\ref{k2:entstehungHalbleiter} Enstehung der Halbleiter &&&&&&&& X \\
\ref{k2:phononen} Phononen &&&&&&&& X \\
\ref{k3:diffusion} Berechnung Diffusionsspannung & X &&&&&&& X \\
\ref{k3:alleGleichungen} Alle Gleichungen & X &&&&&&& \\
\ref{k3:dotieren} Dotieren &&&&&&&& X \\
\ref{k3:ladungstraegerkonz} Verlauf der Ladungstr\"agerkonzentration &&&&&&&& X \\
\ref{k3:drude} Drude Modell &&&&&&&& X \\
\ref{k3:halleffekt} Hall-Effekt &&&&&&&& X \\
\ref{k3:hallspannung} Hall-Spannung &&&&&&&& X \\
\ref{k3:diffusionsstrom} Diffusionsstrom &&&&&&&& X \\
\ref{k3:stromgleichungen} Stromgleichungen, J\_n, J\_p &&&&&&&& X \\
\ref{k3:shockleyhaynes} Shockley-Haynes Experiment &&&&&&&& X \\
\ref{k3:kontinuitaet} Kontinuit\"atsgleichungen &&&&&&&& X \\
\ref{k4:laser} Wie funktionieren Laser & X &&&&&&& X\\
\ref{k4:inUndIndirekt} Direkter und Indirekter Halbleiter&&& X &&&&& X \\
\ref{k4:zustandsDichte} Zustandsdichte und Besetzungswahrscheinlichkeit && X &&&&& X \\
\ref{k5:pn} PN-\"Ubergang && X & X & X &X&X&& X \\
\ref{k5:diode} Diodenkennlinie &&&&&&&& X \\
\ref{k5:pnBand} Banddiagramm f\"ur PN ohne Spannung  && X &&&&&& \\
\ref{k5:tunnelDiode} Tunneldiode&&& X && X & X & X & X \\
\ref{k5:tunnelEffekt} Tunneln Allgemein&&&&& X &&& X\\
\ref{k5:schottky} Schottky-Kontakt-Diode &&&&&&&& X \\
\ref{k5:backward} Backward-Diode &&&&&&&& X \\
\ref{k5:heterostrukturen} Heterostrukturen &&&&&&&& X \\
\ref{k6:bipolar} Funktion von Bipolartransistoren & X &&&&&&& X \\
\ref{k6:mosInversion} MOS-Struktur & X && X &&&&& \\
\ref{k6:diffusionsdreieck} Diffusions-Dreieck beim Transistor &&&&&&&& X \\
\ref{k6:fet} Feldeffekt-Transistor &&&&&&&& X \\
\ref{k6:mosfet} MOS-FET &&&&&&&& X \\
\ref{k6:mesfet} MES-FET &&&&&&&& X \\
\ref{k6:early} Early-Effekt &&&&&&&& X \\
\ref{k6:jfet} JFET &&&&&&&& X \\

\hline

\hline

\end{tabular}
\end{adjustbox}
%\end{adjustwidth}
\caption{Mündliche Prüfungsfragen aus diversen Aufzeichnungen der FET.}
\end{table}
\end{center}

%--------------------------------------------------------------------------------
%--------------------------------------------------------------------------------
%--------------------------------------------------------------------------------
\section{Physikalische Grundlagen - Kapitel 1}
\subsection{Photo-Effekt \todo{0x}}\label{k1:photoEf}
Austrittsarbeit?
\subsection{Compton-Effekt \todo{0x}}\label{k1:comptonEf}
Mit welchen 2 Gleichungen l\"asst sich der Compton-Effekt erkl\"aren? Energie und Impulserhaltung
\subsection{Dispersionsrelation \todo{4x}}\label{k1:dispersionsrelation}
\subsubsection{Bandstruktur}
\subsubsection{Bandlücke}
\subsubsection{direkt/ indirket}
\subsubsection{Effektive Masse (was ist schwerer, Elektron im Leitungsband oder Loch im Valenzband?)}
\subsubsection{Wo liegt k=0, E=0 für dirket/indirekt}
\subsubsection{wo ist Elektron nach anheben ins Leitungsband bei indirektem HL (energetisch günstig!)}
\subsubsection{Gedankenexperiment: wenn Elektron in Leitungsband, und Valenzband voll besetzt, was passiert, wenn man HL abkühlt?}

\subsubsection{Zusammenhang zwischen Energie E und der Kreiswellenzahl k}
\subsubsection{Lösung für freies Teilchen}
\subsubsection{Dispersion im Halbleiter}

\subsubsection{Skizze Valenzband, Leitungsband, Ferminiveau}
\subsubsection{Direkter indirekter Halbleiter}
\subsubsection{E(k) Diagramm einzeichen k = 0 , E = 0}

\subsubsection{Elektronen und Löcher}
Was bewegt sich wie, Loch zu Elektron, Elektron zu Loch

\subsubsection{Effektive Masse, Definitionen, Masssen}


\subsection{Schrödinger Gleichung \todo{3x}}\label{k1:schrGl}
\subsubsection{Herleitung für freies Teilchen}
Basiert auf der Annahme dass Welle - Teilchen - Dualismus exisiert. 

Dann gilt nämlich sowohl $E = \hbar \omega$ als auch $p = \hbar * k$

\subsubsection{Spezialfall der Schrödingergleichung für ein \textit{freies} Teilchen:}
\begin{equation}
    -\frac{\hbar^2}{2m} \times \frac{\partial^2}{\partial x^2} \Psi(x,t)  = i \hbar \times \frac{\partial}{\partial t} \Psi(x,t)
\end{equation}

\subsubsection{Allgemeine Form der Schrödingergleichung für ein Teilchen:}

\paragraph{Für den eindimensionalen Fall:}
\begin{equation}
    -\frac{\hbar^2}{2m} \times \frac{\partial^2}{\partial x^2} \Psi(x,t) + V(x,t) \times \Psi(x,t) = i \hbar \times \frac{\partial}{\partial t} \Psi(x,t)
\end{equation}

\paragraph{Für den dreidimensionalen Fall:}

Einfach alles durch Vektoren austauschen mit: 

\begin{equation}
    \frac{\partial^2}{\partial x^2} \longrightarrow \frac{\partial^2}{\partial x^2} + \frac{\partial^2}{\partial y^2} + \frac{\partial^2}{\partial z^2} \equiv \Delta
\end{equation}
und 
\begin{equation}
    E = \frac{\vec{p}^2}{2m} + V(\vec{x})
\end{equation}
was dann schließlich einfach

\begin{equation}
    -\frac{\hbar^2}{2m} \times \Delta \Psi(\vec{x},t) + V(\vec{x},t) \times \Psi(\vec{x},t) = i \hbar \times \frac{\partial}{\partial t} \Psi(\vec{x},t)
\end{equation}
ergibt. Man sieht, es ändert sich eigentlich nicht viel - einfach aus dem Ort \textit{x} und dem Impuls \textit{p} einen Vektor machen und logisch durchgehen wo diese ersetzt werden müssen. 

\begin{enumerate}
    \item Warum kann das Potential nicht von der Zeit abhängen
\end{enumerate}

\subsubsection{Dispersionsrelation für ein freies Teilchen}

\subsubsection{Aufstellen der Zeitabhängingen und Zeitunabhängigen Lösung}

\subsubsection{Einzelne Terme erklären}
\begin{enumerate}
    \item Was ist Phi
    \item Welche Abhängigkeiten (x, t) warum
    \item Was ist das Potenzial?
    \item Von was ist das Potenzial abhängig (x, t)
    \item Freies Teilchen in der Schrödingergleichung für V = 0
    \item Lösung der Schrödingergleichung für ein freies Teilchen im E(K) Diagramm
\end{enumerate}

\subsection{e(k) Diagramm \todo{0x}}\label{k1:ekdiag}

\subsection{Unendlich tiefer Potentialtopf \todo{1x}}\label{k1:pottopf}

\subsection{Tunnel-Effekt \todo{1x}}\label{k1:tunnEf}

%--------------------------------------------------------------------------------
%--------------------------------------------------------------------------------
%--------------------------------------------------------------------------------
\section{Periodische Festkörperstrukturen - Kapitel 2}
\subsection{Unterschied Halbleiter vs Metalle \todo{0x}}\label{k2:metalle}
B\"anderschema aufzeichnen

\subsection{Grund für die Bildung von Festkörpern \todo{1x}}\label{k2:festkorper}
Wichtig: Diagramm f\"ur Energie in Abh\"angigkeit der Atomdistanz. Erkl\"arung mit Hilfe der Modellsubstanz H-Atom: was passiert wenn man 2 H-Atome n\"aher bringt?

\subsection{Leitungsband, Valenzband, Ef aufzeichnen \todo{2x}}\label{k2:leitungsBand}

\subsection{Kroning Penny Modell \todo{0x}}\label{k2:kroningpenny}
Erkl\"aren, Skizzen, Rechenvorgang, grafische L\"osung, Herleitung e(k) Diagramm

\subsection{Entstehung der Halbleiter \todo{0x}}\label{k2:entstehungHalbleiter}

\subsection{Phononen \todo{0x}}\label{k2:phononen}

%--------------------------------------------------------------------------------
%--------------------------------------------------------------------------------
%--------------------------------------------------------------------------------
\section{Transporteffekte in Halbleitern - Kapitel 3}

\subsection{Berechnung der Diffusionsspannung \todo{1x}}\label{k3:diffusion}
Feldst\"arke (Poissongleichung) integrieren und die Skizzen von pn-\"Ubergang, elektrischer
Feldst\"arke und Spannungsverlauf.

\subsection{Alle Gleichungen (2x Stromgleichung, 2x Kontinuitätsgleichung, 1x Poisson-Gleichung) \todo{1x}}\label{k3:alleGleichungen}

\subsection{Dotieren \todo{0x}}\label{k3:dotieren}

\subsection{Verlauf der Ladungstr\"agerkonzentration als Funktion der Temperatur \todo{0x}}\label{k3:ladungstraegerkonz}

\subsection{Drude Modell \todo{0x}}\label{k3:drude}

\subsection{Hall-Effekt \todo{0x}}\label{k3:halleffekt}
Metallplatte aufzeichnen mit Koordinatensystem und I, B, und F bzw. E als Vektoren.
Kurz beschreiben was im p-HL und n-HL passiert.
Erk\"aren warum der Strom nur in eine Richtung flie{\ss}en kann.

\subsection{Hall-Spannung \todo{0x}}\label{k3:hallspannung}

\subsection{Diffusionsstrom \todo{0x}}\label{k3:diffusionsstrom}

\subsection{Stromgleichungen \todo{0x}}\label{k3:stromgleichungen}

\subsection{Shockley-Haynes Experiment \todo{0x}}\label{k3:shockleyhaynes}
    \subsubsection{Schlatung aufzeichnen} Oszi und Spannungsquelle nicht vergessen
    \subsubsection{Kurve aufzeichnen und erkl\"aren was abgelesen werden kann}

\subsection{Kontinuit\"atsgleichungen \todo{0x}}\label{k3:kontinuitaet}

%--------------------------------------------------------------------------------
%--------------------------------------------------------------------------------
%--------------------------------------------------------------------------------
\section{Optische Eigenschaften von Halbleitern - Kapitel 4}

\subsection{Laser \todo{1x}}\label{k4:laser}
    \subsubsection{Wie funktionieren HL-Laser?}
    \subsubsection{E(k) Diagramm}
    \subsubsection{Inversion}
    \subsubsection{stimulierte Emission}
    \subsubsection{Verst\"arkerbedingung}

\subsection{Vergleiche direkte und indirekte Halbleiter \todo{1x}}\label{k4:inUndIndirekt}
    \subsubsection{Wo ist k=0, was bedeutet das?}
    \subsubsection{w(k) für freies Teilchen}
    \subsubsection{Bewegt sich ein e- im Ev bei T=0?}
    \subsubsection{Wo ist die Masse am größten (Ev, Ec)?}
    \subsubsection{Gleichung für m*}

Banddiagramm, Band\"ubergang
\subsection{Zustandsdichte und Besetzungswahrscheinlichkeit \todo{2x}}\label{k4:zustandsDichte}
    \subsubsection{Banddiagramme}
    \subsubsection{Temperaturabh\"angigkeit}

%--------------------------------------------------------------------------------
%--------------------------------------------------------------------------------
%--------------------------------------------------------------------------------
\section{Halbleiterdioden - Kapitel 5}
\subsection{PN-Übergang \todo{5x}}\label{k5:pn}
	\subsubsection{aufzeichnen ohne externe Spannung}
	%\subsubsection{Diagramm dazu, wo auf x-Achse Ort (Verlauf von pn Übergang) und auf der y-Achse Dichte der freien Elektronen}
	%(Achtung: Raumladungszone! Auf Größenordnungen achten!)
	
	%\subsubsection{Größenordnung einschätzen: Abstand von Elektronen, Größe von RLZ (normaler PN Übergang, Tunneldiode), Elektronendichte im HL}
	
	\subsubsection{Skizze Flussrichtung, Sperrichtung, Raumladungszone}
	\subsubsection{Raumladungszone}
	\subsubsection{Skalen, logarithmisch, linear, Größenordnung}
	\subsubsection{Freie Elektronen}
	\subsubsection{Raumladungszone}
	\subsubsection{Anzahl}
	\subsubsection{Dotierung}
	
	\subsubsection{Intrinsische Ladungsträgerdichte ni + Gleichung $n*p=ni^2$ }
	\subsubsection{ni ist Funktion von was? (T und Gap)}
	
\subsection{Diodenkennlinie \todo{0x}}\label{k5:diode}
    \subsubsection{Warum exponentiell?}
    \subsubsection{Verlauf der Kapazit\"at einer Diode als Funktion der Spannung}
    \subsubsection{Wie sehen die Ladungen bei anlegen einer Spannung aus?}
    \subsubsection{Massenwirkungsgesetz}

\subsection{Banddiagramm für PN ohne Spannung \todo{1x}}\label{k5:pnBand}

\subsection{Tunneldiode \todo{4x}}\label{k5:tunnelDiode}
    \subsubsection{Banddiagramm ohne externe Spannung}
    \subsubsection{was passiert, wenn 25mV in Flussrichtung angelegt werden?}
    \subsubsection{im Banddiagramm Konstellation aufzeichnen, wo Maximum und Minimum in Kennlinie auftreten}
    
    \subsubsection{Dicke Raumladungszone Dotierung}
    \subsubsection{Wie groß ist die Bandlücke in Volt}
    \subsubsection{Kennlinie der Tunneldiode}
    \subsubsection{Kennlinie einer normalen Diode}
    
    \subsubsection{Ferminiveau (Skizze)}
    \subsubsection{Valenzband}
    \subsubsection{Leitungsband}
    \subsubsection{Ef aufzeichnen}

    \subsubsection{Warum Tunneldiode? Was tunnelt wo und wann? Was erhöht die Wahrscheinlichkeit?}

\subsection{Tunneleffekt \todo{1x}}\label{k5:tunnelEffekt}
    \subsubsection{Flussspannung}
    \subsubsection{Energieerhaltung}
    \subsubsection{Endliche Barriere}
    \subsubsection{Einfaches Tunneln}

\subsection{Schottky-Kontakt-Diode \todo{0x}}\label{k5:schottky}

\subsection{Backward-Diode \todo{0x}}\label{k5:backward}

\subsection{Heterostrukturen \todo{0x}}\label{k5:heterostrukturen}


%--------------------------------------------------------------------------------
%--------------------------------------------------------------------------------
%--------------------------------------------------------------------------------
\section{Transistoren - Kapitel 6}
\subsection{Funktion von Bipolartransistoren \todo{1x}}\label{k6:bipolar}
    \subsubsection{Funktionionsweise}
    \subsubsection{Skizze}
    \subsubsection{Warum wird er als Verst\"arker eingesetzt?}
\subsection{MOS-Struktur und Inversion \todo{2x}}\label{k6:mosInversion}

    \subsubsection{Wann spricht man von Inversion}
    \subsubsection{Leitungsband, Valenzband, Ef aufzeichnen}

\subsection{Diffusions-Dreieck beim Transistor \todo{0x}}\label{k6:diffusionsdreieck}
    \subsubsection{Warum fast linear?} exponential-Funktion im Ursprung ann\"ahernd linear
    \subsubsection{Was w\"are wenn die Basisl\"ange gr\"o{\ss}er als die Diffusionsl\"ange w\"are?} Man h\"atte die Wirkung von 2 Dioden und keinen Transistoreffekt mehr.

\subsection{Feldeffekt-Transistor \todo{0x}}\label{k6:fet}

\subsection{MOS-FET \todo{0x}}\label{k6:mosfet}
    \subsubsection{Aufbau}
    \subsubsection{Bandstruktur}

\subsection{MES-FET \todo{0x}}\label{k6:mesfet}

\subsection{Early-Effekt \todo{0x}}\label{k6:early}

\subsection{JFET \todo{0x}}\label{k6:jfet}

%--------------------------------------------------------------------------------
%--------------------------------------------------------------------------------
%--------------------------------------------------------------------------------
\section{Ohne Kapitel}
\end{document}
