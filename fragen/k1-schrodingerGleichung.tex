\subsubsection{Herleitung für freies Teilchen}
Basiert auf der Annahme dass Welle - Teilchen - Dualismus exisiert. 

Dann gilt nämlich sowohl $E = \hbar \omega$ als auch $p = \hbar * k$

\subsubsection{Spezialfall der Schrödingergleichung für ein \textit{freies} Teilchen:}
\begin{equation}
    -\frac{\hbar^2}{2m} \times \frac{\partial^2}{\partial x^2} \Psi(x,t)  = i \hbar \cdot \frac{\partial}{\partial t} \Psi(x,t)
\end{equation}

\subsubsection{Allgemeine Form der Schrödingergleichung für ein Teilchen:}

\paragraph{Für den eindimensionalen Fall:}
\begin{equation}
    -\frac{\hbar^2}{2m} \cdot \frac{\partial^2}{\partial x^2} \Psi(x,t) + V(x,t) \cdot \Psi(x,t) = i \hbar \times \frac{\partial}{\partial t} \Psi(x,t)
\end{equation}

\paragraph{Für den dreidimensionalen Fall:}

Einfach alles durch Vektoren austauschen mit: 

\begin{equation}
    \frac{\partial^2}{\partial x^2} \longrightarrow \frac{\partial^2}{\partial x^2} + \frac{\partial^2}{\partial y^2} + \frac{\partial^2}{\partial z^2} \equiv \Delta
\end{equation}
und 
\begin{equation}
    E = \frac{\vec{p}^2}{2m} + V(\vec{x})
\end{equation}
was dann schließlich einfach

\begin{equation}
    -\frac{\hbar^2}{2m} \cdot \Delta \Psi(\vec{x},t) + V(\vec{x},t) \cdot \Psi(\vec{x},t) = i \hbar \cdot \frac{\partial}{\partial t} \Psi(\vec{x},t)
\end{equation}
ergibt. Man sieht, es ändert sich eigentlich nicht viel - einfach aus dem Ort \textit{x} und dem Impuls \textit{p} einen Vektor machen und logisch durchgehen wo diese ersetzt werden müssen. 

\begin{enumerate}
    \item Warum kann das Potential nicht von der Zeit abhängen
\end{enumerate}

\subsubsection{Dispersionsrelation für ein freies Teilchen}

\subsubsection{Aufstellen der Zeitabhängingen und Zeitunabhängigen Lösung}

\subsubsection{Einzelne Terme erklären}
\begin{enumerate}
    \item Was ist Phi
    \item Welche Abhängigkeiten (x, t) warum
    \item Was ist das Potenzial?
    \item Von was ist das Potenzial abhängig (x, t)
    \item Freies Teilchen in der Schrödingergleichung für V = 0
    \item Lösung der Schrödingergleichung für ein freies Teilchen im E(K) Diagramm
\end{enumerate}