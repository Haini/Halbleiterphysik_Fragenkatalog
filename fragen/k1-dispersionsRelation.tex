Die Dispersionsrelation $\omega(k)$ beschreibt den Zusammenhang zwischen Zeit- und Ortsabh\"angigkeit.
Sie h\"angt vom Medium ab, in dem sich die Wellen ausbreiten.

\subsubsection{Bandstruktur}
\subsubsection{Bandlücke}
\subsubsection{direkt/ indirket}
\subsubsection{Effektive Masse (was ist schwerer, Elektron im Leitungsband oder Loch im Valenzband?)}
\subsubsection{Wo liegt k=0, E=0 für dirket/indirekt}
\subsubsection{wo ist Elektron nach anheben ins Leitungsband bei indirektem HL (energetisch günstig!)}
\subsubsection{Gedankenexperiment: wenn Elektron in Leitungsband, und Valenzband voll besetzt, was passiert, wenn man HL abkühlt?}

\subsubsection{Zusammenhang zwischen Energie E und der Kreiswellenzahl k}
\subsubsection{Lösung für freies Teilchen}
\subsubsection{Dispersion im Halbleiter}

\subsubsection{Skizze Valenzband, Leitungsband, Ferminiveau}
\subsubsection{Direkter indirekter Halbleiter}
\subsubsection{E(k) Diagramm einzeichen k = 0 , E = 0}

\subsubsection{Elektronen und Löcher}
Was bewegt sich wie, Loch zu Elektron, Elektron zu Loch

\subsubsection{Effektive Masse, Definitionen, Masssen}