\documentclass{article}
\usepackage[utf8]{inputenc}

\usepackage{tabularx}
\usepackage{graphicx}
\usepackage{adjustbox}
\usepackage{hyperref} 

\title{Halbleiterphysik Fragenkatalog}
\author{Constantin Schieber\\Benedikt Tutzer}
\date{August 2020}

\begin{document}

\maketitle
\vfill
Diese Ausarbeitung ist zur Vorbereitung auf die m\"undliche Pr\"ufung im August 2020 entstanden. Sie kann gern auf overleaf aktuell gehalten werden.\\

\begin{center}\href{Overleaf public link}{https://www.overleaf.com/4852512811wdbqyxrbkgyb}\end{center}

\setcounter{section}{-1}
\newpage
\tableofcontents
\newpage

%--------------------------------------------------------------------------------
%--------------------------------------------------------------------------------
%--------------------------------------------------------------------------------
\section{Overview}

\begin{table}[h]

\begin{adjustbox}{width=\textwidth}
\begin{tabular}{lcccccccc}
Frage                & 02.16 & 01.28 & 02.19 & 03.04 & 03.05 & 03.21 & 01.16 & Katalog\\
                     & 2010  & 2014  & 2019  & 2019  & 2019  & 2019  & 2020 & \\
                     \hline
\ref{k1:schrGl} Schrödinger Gleichung&&&& X &  & X & X &  \\
\ref{k1:tunnEf} Tunneleffekt&&&&&&&X & \\
\ref{k2:festkorper} Grund für Festkörperbildung & X &&&&&&& \\
\ref{k4:laser} Wie funktionieren Laser & X &&&&&&& \\
\ref{k4:inUndIndirekt} Direkter und Indirekter Halbleiter&&& X &&&&& \\
\ref{k5:pn} PN-\"Ubergang && X & X & X &X&X&& \\
\ref{k5:pnBand} Banddiagramm f\"ur PN ohne Spannung  && X &&&&&& \\
\ref{k5:tunnelDiode} Tunneldiode&&& X && X & X & X & \\
\ref{k5:tunnelEffekt} Tunneln Allgemein&&&&& X &&& \\
\ref{k6:bipolar} Funktion von Bipolartransistoren & X &&&&&&& \\
\ref{k6:mosInversion} MOS-Inversion & X &&&&&&& \\
\ref{k6:mosInversion} MOS-Struktur(feat Inversion)&&& X &&&&& \\
\ref{kX:alleGleichungen} Alle Gleichungen & X &&&&&&& \\
\ref{kX:diffusion} Berechnung Diffusionsspannung & X &&&&&&& \\
\ref{kX:leitungsBand} Leitungsband, Valenzband, Ef aufzeichnen && X & X &&&&& \\
\ref{kX:zustandsDichte} Zustandsdichte&& X &&&&& \\
\ref{kX:dispersionsrelation} Dispersionsrelation für Halbleiter && X &&& X & X & X & \\

\hline

\hline
\end{tabular}
\end{adjustbox}
\end{table}

%--------------------------------------------------------------------------------
%--------------------------------------------------------------------------------
%--------------------------------------------------------------------------------
\section{Physikalische Grundlagen - Kapitel 1}
\subsection{Photo-Effekt}\label{k1:photoEf}
\subsection{Compton-Effekt}\label{k1:comptonEf}
\subsection{Dispersionsrelation}\label{kX:dispersionsrelation}
\subsubsection{Bandstruktur}
\subsubsection{Bandlücke}
\subsubsection{direkt/ indirket}
\subsubsection{Effektive Masse (was ist schwerer, Elektron im Leitungsband oder Loch im Valenzband?)}
\subsubsection{Wo liegt k=0, E=0 für dirket/indirekt}
\subsubsection{wo ist Elektron nach anheben ins Leitungsband bei indirektem HL (energetisch günstig!)}
\subsubsection{Gedankenexperiment: wenn Elektron in Leitungsband, und Valenzband voll besetzt, was passiert, wenn man HL abkühlt?}

\subsubsection{Zusammenhang zwischen Energie E und der Kreiswellenzahl k}
\subsubsection{Lösung für freies Teilchen}
\subsubsection{Dispersion im Halbleiter}

\subsubsection{Skizze Valenzband, Leitungsband, Ferminiveau}
\subsubsection{Direkter indirekter Halbleiter}
\subsubsection{E(k) Diagramm einzeichen k = 0 , E = 0}

\subsubsection{Elektronen und Löcher}
Was bewegt sich wie, Loch zu Elektron, Elektron zu Loch

\subsubsection{Effektive Masse, Definitionen, Masssen}


\subsection{Schrödinger Gleichung}\label{k1:schrGl}
\subsubsection{Herleitung für freies Teilchen}
Basiert auf der Annahme dass Welle - Teilchen - Dualismus exisiert. 

Dann gilt nämlich sowohl $E = \hbar \omega$ als auch $p = \hbar * k$

\subsubsection{Spezialfall der Schrödingergleichung für ein \textit{freies} Teilchen:}
\begin{equation}
    -\frac{\hbar^2}{2m} \times \frac{\partial^2}{\partial x^2} \Psi(x,t)  = i \hbar \times \frac{\partial}{\partial t} \Psi(x,t)
\end{equation}

\subsubsection{Allgemeine Form der Schrödingergleichung für ein Teilchen:}

\paragraph{Für den eindimensionalen Fall:}
\begin{equation}
    -\frac{\hbar^2}{2m} \times \frac{\partial^2}{\partial x^2} \Psi(x,t) + V(x,t) \times \Psi(x,t) = i \hbar \times \frac{\partial}{\partial t} \Psi(x,t)
\end{equation}

\paragraph{Für den dreidimensionalen Fall:}

Einfach alles durch Vektoren austauschen mit: 

\begin{equation}
    \frac{\partial^2}{\partial x^2} \longrightarrow \frac{\partial^2}{\partial x^2} + \frac{\partial^2}{\partial y^2} + \frac{\partial^2}{\partial z^2} \equiv \Delta
\end{equation}
und 
\begin{equation}
    E = \frac{\vec{p}^2}{2m} + V(\vec{x})
\end{equation}
was dann schließlich einfach

\begin{equation}
    -\frac{\hbar^2}{2m} \times \Delta \Psi(\vec{x},t) + V(\vec{x},t) \times \Psi(\vec{x},t) = i \hbar \times \frac{\partial}{\partial t} \Psi(\vec{x},t)
\end{equation}
ergibt. Man sieht, es ändert sich eigentlich nicht viel - einfach aus dem Ort \textit{x} und dem Impuls \textit{p} einen Vektor machen und logisch durchgehen wo diese ersetzt werden müssen. 

\begin{enumerate}
    \item Warum kann das Potential nicht von der Zeit abhängen
\end{enumerate}

\subsubsection{Dispersionsrelation für ein freies Teilchen}

\subsubsection{Aufstellen der Zeitabhängingen und Zeitunabhängigen Lösung}

\subsubsection{Einzelne Terme erklären}
\begin{enumerate}
    \item Was ist Phi
    \item Welche Abhängigkeiten (x, t) warum
    \item Was ist das Potenzial?
    \item Von was ist das Potenzial abhängig (x, t)
    \item Freies Teilchen in der Schrödingergleichung für V = 0
    \item Lösung der Schrödingergleichung für ein freies Teilchen im E(K) Diagramm
\end{enumerate}

\subsection{Tunnel-Effekt}\label{k1:tunnEf}
\subsection{Unendlich tiefer Potentialtopf}\label{k1:potentialtopf}


%--------------------------------------------------------------------------------
%--------------------------------------------------------------------------------
%--------------------------------------------------------------------------------
\section{Periodische Festkörperstrukturen - Kapitel 2}
\subsection{Grund für die Bildung von Festkörpern}\label{k2:festkorper}
\subsection{Leitungsband, Valenzband, Ef aufzeichnen}\label{kX:leitungsBand}

%--------------------------------------------------------------------------------
%--------------------------------------------------------------------------------
%--------------------------------------------------------------------------------
\section{Transporteffekte in Halbleitern - Kapitel 3}

\subsection{Berechnung der Diffusionsspannung}\label{kX:diffusion}
\subsection{Alle Gleichungen (2x Stromgleichung, 2x Kontinuitätsgleichung, 1x Poisson-Gleichung)}\label{kX:alleGleichungen}

%--------------------------------------------------------------------------------
%--------------------------------------------------------------------------------
%--------------------------------------------------------------------------------
\section{Optische Eigenschaften von Halbleitern - Kapitel 4}

\subsection{Laser}\label{k4:laser}
\subsection{Vergleiche direkte und indirekte Halbleider}\label{k4:inUndIndirekt}
\subsection{Zustandsdichte}\label{kX:zustandsDichte}


%--------------------------------------------------------------------------------
%--------------------------------------------------------------------------------
%--------------------------------------------------------------------------------
\section{Halbleiterdioden - Kapitel 5}
\subsection{PN-Übergang}\label{k5:pn}
	\subsubsection{aufzeichnen ohne externe Spannung}
	\subsubsection{Diagramm dazu, wo auf x-Achse Ort (Verlauf von pn Übergang) und auf der y-Achse Dichte der freien Elektronen}
	(Achtung: Raumladungszone! Auf Größenordnungen achten!)
	\subsubsection{Größenordnung einschätzen: Abstand von Elektronen, Größe von RLZ (normaler PN Übergang, Tunneldiode), Elektronendichte im HL}

\subsection{Banddiagramm für PN ohne Spannung}\label{k5:pnBand}
\subsection{Tunneldiode}\label{k5:tunnelDiode}
	\subsubsection{Banddiagramm ohne externe Spannung}
	\subsubsection{Tunneleffekt erklären}
	\subsubsection{was passiert, wenn 25mV in Flussrichtung angelegt werden?}
	\subsubsection{Kennlinie aufzeichnen}
	\subsubsection{im Banddiagramm Konstellation aufzeichnen, wo Maximum und Minimum in Kennlinie auftreten}
\subsection{Tunneleffekt}\label{k5:tunnelEffekt}

%--------------------------------------------------------------------------------
%--------------------------------------------------------------------------------
%--------------------------------------------------------------------------------
\section{Transistoren - Kapitel 6}
\subsection{Funktion von Bipolartransistoren}\label{k6:bipolar}
\subsection{MOS-Inversion}\label{k6:mosInversion}

%--------------------------------------------------------------------------------
%--------------------------------------------------------------------------------
%--------------------------------------------------------------------------------
\section{Ohne Kapitel}
\end{document}
